%==============================================================================
\section{Problem statement }
%==============================================================================


In the current paradigm of cosmology the universe 
has several components of energy and matter, but one is taken as the responsible for the accelerated expansion
observed,  dark energy (DE). 
According to recent estimations DE accounts  for $\sim 70 \%$ of the total density of the universe. This highlights the importance
of understanding the nature of DE including the state equations of it.  In this case, it is essential to know the hole picture, i.e., 
to study the amount of DE in different epochs of the universe, even at high redshifts. Therefore, a standard ruler for high redshifts 
becomes necessary and the baryonic acoustic oscillations appear as an excellent choice. 

\

The early universe was homogeneous and isotropic except for small dark matter perturbations. In this scenario
the radiation and the baryonic matter were coupled thanks to the high temperatures. Due to the contrast density 
the baryonic matter was attracted gravitationally,  but because of the coupling with radiation baryonic oscillations 
were caused. The universe continue expanding causing temperature to drop and this way decoupling occurred. 
Then, there were no more radiation pressure affecting baryonic matter and the velocity of oscillations gradually dropped 
leading an imprint in the clustering of the universe. The scale of the imprint is generally called the sound horizon. 
Therefore, not only initial dark matter perturbations caused baryonic 
and dark matter to fall in but also the imprint caused dark matter to fall in this baryonic density perturbations. 
A way to observe this preferred scale is through the power spectrum or the correlation function either an observation 
or a cosmological simulation. 

\ 

But, the power spectrum obtained from observations let several problems in the baryonic acoustic oscillations
recovered, because of the non linear collapse, redshift biases, the dependence on the target selection of the observation, 
moreover the relation between the galaxy power spectrum and dark matter spectrum. Hence, a comparison 
with the power spectrum constructed from cosmological simulations not only turns out to be a useful but 
necessary step to study and constraint the deviations in the shape and position of the baryonic acoustic 
oscillations due to aspects exposed before. 

\

Observational measurements of baryonic acoustic oscillations have been done in several previous works such 
as \cite{Obs01}, \cite{Obs02}, \cite{Obs03}, \cite{Obs04} . Measurements of baryonic acoustic oscillations on simulations 
have also been done in these papers \cite{Sim01}, \cite{Sim02},\cite{Sim03}, \cite{Sim04}.
And theoretical studies of baryonic acoustic oscillation using non linear theory have been realized in \cite{Theo01}, \cite{Theo02},
\cite{Theo03}, \cite{last}  too.  
In the present work, there is going to be done a comparison between the power spectrum of cosmological 
simulations and the one obtained from observations of the Sloan Digital Sky Survey (SDSS). 
The method to construct the power spectrum for the cosmological simulations is shown in \ref{subsubsec:CPS}. 
The method to construct the density field for the observations is the halo based method proposed by \cite{HBM} (section \ref{subsubsec:HBM}), the 
advantage it presents is that allows to make reconstructions of the mass distribution without making strong 
supositions on the bias between dark and baryonic matter. The halo based method will also allow to analyse how 
the structure scale is related to the amplitude and width of the baryon acoustic oscillations. 


