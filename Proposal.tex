\documentclass[a4,useAMS,usenatbib,usegraphicx,12pt]{article}
%External Packages and personalized macros
%=========================================================================
%		EXTERNAL PACKAGES
%=========================================================================
\usepackage[round]{natbib}
\usepackage[margin=3cm]{geometry}
\usepackage{hyperref}
\usepackage{times}
\usepackage{amsmath} 
\usepackage{amssymb}
\usepackage{graphicx}
\usepackage{array, xcolor, lipsum, bibentry}
\usepackage[nottoc, notlof, notlot]{tocbibind}
%\usepackage[spanish, activeacute]{babel} 
%\usepackage[utf8]{inputenc}
   
\definecolor{lightgray}{gray}{0.8}
\newcolumntype{L}{>{\raggedleft}p{0.14\textwidth}}
\newcolumntype{R}{p{0.8\textwidth}}
\newcommand\VRule{\color{lightgray}\vrule width 0.5pt}

\usepackage{booktabs}% http://ctan.org/pkg/booktabs
\newcommand{\tabitem}{~~\llap{\textbullet}~~}

%=========================================================================
%		INTERNAL MACROS
%=========================================================================
% To highlight comments 
\definecolor{red}{rgb}{1,0.0,0.0}
\newcommand{\red}{\color{red}}
\definecolor{darkgreen}{rgb}{0.0,0.5,0.0}
\newcommand{\SRK}[1]{\textcolor{darkgreen}{\bf SRK: \textit{#1}}}
\newcommand{\SRKED}[1]{\textcolor{darkgreen}{\bf #1}}

\newcommand{\VPH}{\texttt{VPH}}
\newcommand{\SPH}{\texttt{SPH}}
\newcommand{\AMR}{\texttt{AMR}}
\newcommand{\AREPO}{\texttt{AREPO}}
\newcommand{\VTFE}{\texttt{VTFE}}

\newcommand{\LCDM}{$\Lambda$CDM~}
\newcommand{\beq}{\begin{eqnarray}}  
\newcommand{\eeq}{\end{eqnarray}}  
\newcommand{\zz}{$z\sim 3$} 
\newcommand{\apj}{ApJ}  
\newcommand{\apjs}{ApJS}  
\newcommand{\apjl}{ApJL}  
\newcommand{\aj}{AJ}  
\newcommand{\mnras}{MNRAS}  
\newcommand{\mnrassub}{MNRAS accepted}  
\newcommand{\aap}{A\&A}  
\newcommand{\aaps}{A\&AS}  
\newcommand{\araa}{ARA\&A}  
\newcommand{\nat}{Nature}  
\newcommand{\physrep}{PhR}
\newcommand{\pasp}{PASP}    
\newcommand{\pasj}{PASJ}    
\newcommand{\avg}[1]{\langle{#1}\rangle}  
\newcommand{\ly}{{\ifmmode{{\rm Ly}\alpha}\else{Ly$\alpha$}\fi}}
\newcommand{\hMpc}{{\ifmmode{h^{-1}{\rm Mpc}}\else{$h^{-1}$Mpc }\fi}}  
\newcommand{\hGpc}{{\ifmmode{h^{-1}{\rm Gpc}}\else{$h^{-1}$Gpc }\fi}}  
\newcommand{\hmpc}{{\ifmmode{h^{-1}{\rm Mpc}}\else{$h^{-1}$Mpc }\fi}}  
\newcommand{\hkpc}{{\ifmmode{h^{-1}{\rm kpc}}\else{$h^{-1}$kpc }\fi}}  
\newcommand{\hMsun}{{\ifmmode{h^{-1}{\rm {M_{\odot}}}}\else{$h^{-1}{\rm{M_{\odot}}}$}\fi}}  
\newcommand{\hmsun}{{\ifmmode{h^{-1}{\rm {M_{\odot}}}}\else{$h^{-1}{\rm{M_{\odot}}}$}\fi}}  
\newcommand{\Msun}{{\ifmmode{{\rm {M_{\odot}}}}\else{${\rm{M_{\odot}}}$}\fi}}  
\newcommand{\msun}{{\ifmmode{{\rm {M_{\odot}}}}\else{${\rm{M_{\odot}}}$}\fi}}  
\newcommand{\lya}{{Lyman$\alpha$~}}
\newcommand{\clara}{{\texttt{CLARA}}~}
\newcommand{\rand}{{\ifmmode{{\mathcal{R}}}\else{${\mathcal{R}}$ }\fi}}  


%MY COMMANDS #############################################################
\newcommand{\sub}[1]{\mbox{\scriptsize{#1}}}
\newcommand{\dtot}[2]{ \frac{ d #1 }{d #2} }
\newcommand{\dpar}[2]{ \frac{ \partial #1 }{\partial #2} }
\newcommand{\pr}[1]{ \left( #1 \right) }
\newcommand{\corc}[1]{ \left[ #1 \right] }
\newcommand{\lla}[1]{ \left\{ #1 \right\} }
\newcommand{\bds}[1]{\boldsymbol{ #1 }}
\newcommand{\oiint}{\displaystyle\bigcirc\!\!\!\!\!\!\!\!\int\!\!\!\!\!\int}
\newcommand{\mathsize}[2]{\mbox{\fontsize{#1}{#1}\selectfont $#2$}}
\newcommand{\eq}[2]{\begin{equation} \label{eq:#1} #2 \end{equation}}
\newcommand{\lth}{$\lambda_{th}$ }
%#########################################################################

\setlength\parindent{0pt}

 
\title{ {\textbf{Baryon acoustic oscillations in the dark matter halos in the SDSS}}\\ 
				\Large Research Proposal for a Master Thesis in Physics\\ 
				\color{black}\rule{15cm}{0.5mm} }
\author{Nataly Mateus Londono}
\date{}
  
\begin{document}  
\maketitle
%\begin{center}
%\includegraphics[trim = 0mm 3.5cm 0mm 3.0cm, clip, keepaspectratio=true,width=0.7\textwidth,natwidth=610,natheight=642]{./latex/Presentation1.png}
%\tiny{\\ d }
%\end{center}
\hypersetup{linkbordercolor=white}
\tableofcontents
 
\newpage 

%============================================================================== 
\section{General Information}
\small
\subsection*{Information of the Student}
\begin{tabular}{L!{\VRule}R}
\bf Name		& Nataly Mateus Londoño\\
\bf Degree		& B.Sc. in Physics, Universidad de Antioquia \\
\bf Position	& Adjunct Professor, Universidad de Antioquia\\
\bf E-mail	&  nataly.mateus \textit{at} udea.edu.co \\
\end{tabular}

\vspace{10pt}


\vspace{15pt}  

\subsection*{Information of the Project}
\begin{tabular}{L!{\VRule}R}
\bf Title		& \bf Properties of the BAOs from dark matter halos in the SDSS \\
\bf Field		& Cosmology, Astrophysics, Physical Sciences \\
\bf Advisor 1	& Professor Juan Carlos Munoz-Cuartas. Universidad de Antioquia, Colombia.\\
\bf University	& Universidad de Antioquia, Master of Physics program \\
\bf Time Frame	& 2 years \\
\end{tabular}
\normalsize
%==============================================================================

%==============================================================================
\section{Abstract}
%==============================================================================
\newpage


%==============================================================================
\section{Introduction}
%==============================================================================


%==============================================================================
\section{Theoretical Framework}
%==============================================================================


%==============================================================================
\section{Objectives}
%==============================================================================

\subsection*{General Objective}

To study the properties of the baryon accoustic osscillations (BAOs), amplitude and width, using as 
tracers the distribution of the halos in the sloan digital sky survey and their dependence with the tracer 
halo population. 

\subsection*{Specific Objectives}

\begin{enumerate}

\item[-] Determine in which way the structure scale is related to the amplitude and width of the baryon
acoustic oscillations. 

\item[-] Find if there is a change in the BAOs position with the structure scale and quantify it. 

\item[-] Find a possible correlation between the BAOs properties and structure formation. 

\end{enumerate}


%==============================================================================
\section{Methodology}
%==============================================================================

It is going to be used observational data from the SDSS, DR-7 and DR-12, to study the BAOs in the 
observable universe. 
There is going to be used N body cosmological simulations to model the formation process of large 
scale structure. 
The comparision between both of them it will provide information about the BAOs and its properties.  

%==============================================================================
\section{Expected Results}
%==============================================================================


%==============================================================================
\section{Scientific Impact}
%==============================================================================


%==============================================================================
\section{Schedule}
%==============================================================================
	


%==============================================================================
\bibliographystyle{latex/mn2e}
\renewcommand{\bibname}{8\ \ \ \ Bibliography}
\small
\bibliography{references.bib}
%==============================================================================



\end{document}