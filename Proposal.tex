\documentclass[a4,useAMS,usenatbib,usegraphicx,12pt]{article}
%External Packages and personalized macros
%=========================================================================
%		EXTERNAL PACKAGES
%=========================================================================
\usepackage[round]{natbib}
\usepackage[margin=3cm]{geometry}
\usepackage{hyperref}
\usepackage{times}
\usepackage{amsmath} 
\usepackage{amssymb}
\usepackage{graphicx}
\usepackage{array, xcolor, lipsum, bibentry}
\usepackage[nottoc, notlof, notlot]{tocbibind}
%\usepackage[spanish, activeacute]{babel} 
%\usepackage[utf8]{inputenc}
   
\definecolor{lightgray}{gray}{0.8}
\newcolumntype{L}{>{\raggedleft}p{0.14\textwidth}}
\newcolumntype{R}{p{0.8\textwidth}}
\newcommand\VRule{\color{lightgray}\vrule width 0.5pt}

\usepackage{booktabs}% http://ctan.org/pkg/booktabs
\newcommand{\tabitem}{~~\llap{\textbullet}~~}

%=========================================================================
%		INTERNAL MACROS
%=========================================================================
% To highlight comments 
\definecolor{red}{rgb}{1,0.0,0.0}
\newcommand{\red}{\color{red}}
\definecolor{darkgreen}{rgb}{0.0,0.5,0.0}
\newcommand{\SRK}[1]{\textcolor{darkgreen}{\bf SRK: \textit{#1}}}
\newcommand{\SRKED}[1]{\textcolor{darkgreen}{\bf #1}}

\newcommand{\VPH}{\texttt{VPH}}
\newcommand{\SPH}{\texttt{SPH}}
\newcommand{\AMR}{\texttt{AMR}}
\newcommand{\AREPO}{\texttt{AREPO}}
\newcommand{\VTFE}{\texttt{VTFE}}

\newcommand{\LCDM}{$\Lambda$CDM~}
\newcommand{\beq}{\begin{eqnarray}}  
\newcommand{\eeq}{\end{eqnarray}}  
\newcommand{\zz}{$z\sim 3$} 
\newcommand{\apj}{ApJ}  
\newcommand{\apjs}{ApJS}  
\newcommand{\apjl}{ApJL}  
\newcommand{\aj}{AJ}  
\newcommand{\mnras}{MNRAS}  
\newcommand{\mnrassub}{MNRAS accepted}  
\newcommand{\aap}{A\&A}  
\newcommand{\aaps}{A\&AS}  
\newcommand{\araa}{ARA\&A}  
\newcommand{\nat}{Nature}  
\newcommand{\physrep}{PhR}
\newcommand{\pasp}{PASP}    
\newcommand{\pasj}{PASJ}    
\newcommand{\avg}[1]{\langle{#1}\rangle}  
\newcommand{\ly}{{\ifmmode{{\rm Ly}\alpha}\else{Ly$\alpha$}\fi}}
\newcommand{\hMpc}{{\ifmmode{h^{-1}{\rm Mpc}}\else{$h^{-1}$Mpc }\fi}}  
\newcommand{\hGpc}{{\ifmmode{h^{-1}{\rm Gpc}}\else{$h^{-1}$Gpc }\fi}}  
\newcommand{\hmpc}{{\ifmmode{h^{-1}{\rm Mpc}}\else{$h^{-1}$Mpc }\fi}}  
\newcommand{\hkpc}{{\ifmmode{h^{-1}{\rm kpc}}\else{$h^{-1}$kpc }\fi}}  
\newcommand{\hMsun}{{\ifmmode{h^{-1}{\rm {M_{\odot}}}}\else{$h^{-1}{\rm{M_{\odot}}}$}\fi}}  
\newcommand{\hmsun}{{\ifmmode{h^{-1}{\rm {M_{\odot}}}}\else{$h^{-1}{\rm{M_{\odot}}}$}\fi}}  
\newcommand{\Msun}{{\ifmmode{{\rm {M_{\odot}}}}\else{${\rm{M_{\odot}}}$}\fi}}  
\newcommand{\msun}{{\ifmmode{{\rm {M_{\odot}}}}\else{${\rm{M_{\odot}}}$}\fi}}  
\newcommand{\lya}{{Lyman$\alpha$~}}
\newcommand{\clara}{{\texttt{CLARA}}~}
\newcommand{\rand}{{\ifmmode{{\mathcal{R}}}\else{${\mathcal{R}}$ }\fi}}  


%MY COMMANDS #############################################################
\newcommand{\sub}[1]{\mbox{\scriptsize{#1}}}
\newcommand{\dtot}[2]{ \frac{ d #1 }{d #2} }
\newcommand{\dpar}[2]{ \frac{ \partial #1 }{\partial #2} }
\newcommand{\pr}[1]{ \left( #1 \right) }
\newcommand{\corc}[1]{ \left[ #1 \right] }
\newcommand{\lla}[1]{ \left\{ #1 \right\} }
\newcommand{\bds}[1]{\boldsymbol{ #1 }}
\newcommand{\oiint}{\displaystyle\bigcirc\!\!\!\!\!\!\!\!\int\!\!\!\!\!\int}
\newcommand{\mathsize}[2]{\mbox{\fontsize{#1}{#1}\selectfont $#2$}}
\newcommand{\eq}[2]{\begin{equation} \label{eq:#1} #2 \end{equation}}
\newcommand{\lth}{$\lambda_{th}$ }
%#########################################################################

\setlength\parindent{0pt}

 
\title{ {\textbf{Baryon acoustic oscillations in the dark matter halos in the SDSS}}\\ 
				\Large Research Proposal for a Master Thesis in Physics\\ 
				\color{black}\rule{15cm}{0.5mm} }
\author{Nataly Mateus Londono}
\date{}
  
\begin{document}  
\maketitle
%\begin{center}
%\includegraphics[trim = 0mm 3.5cm 0mm 3.0cm, clip, keepaspectratio=true,width=0.7\textwidth,natwidth=610,natheight=642]{./latex/Presentation1.png}
%\tiny{\\ d }
%\end{center}
\hypersetup{linkbordercolor=white}
\tableofcontents
 
\newpage 

%============================================================================== 
\section{General Information}
\small
\subsection*{Information of the Student}
\begin{tabular}{L!{\VRule}R}
\bf Name		& Nataly Mateus Londoño\\
\bf Degree		& B.Sc. in Physics, Universidad de Antioquia \\
\bf Position	& Adjunct Professor, Universidad de Antioquia\\
\bf E-mail	&  nataly.mateus \textit{at} udea.edu.co \\
\end{tabular}

\vspace{10pt}


\vspace{15pt}  

\subsection*{Information of the Project}
\begin{tabular}{L!{\VRule}R}
\bf Title		& \bf Properties of the BAOs from dark matter halos in the SDSS \\
\bf Field		& Cosmology, Astrophysics, Physical Sciences \\
\bf Advisor 1	& Professor Juan Carlos Munoz-Cuartas. Universidad de Antioquia, Colombia.\\
\bf University	& Universidad de Antioquia, Master of Physics program \\
\bf Time Frame	& 2 years \\
\end{tabular}
\normalsize
%==============================================================================

%==============================================================================
\section{Abstract}
%==============================================================================
\newpage


%==============================================================================
\section{Introduction}
%==============================================================================


%==============================================================================
\section{Theoretical Framework}
%==============================================================================


Lo que se puede observar en la actualidad es un universo altamente homogéneo 
e isotrópico a grandes escalas. Además de la radiación cósmica de fondo 
conocemos que en una época más temprana al desacople de la radiación y materia
las inhomogeneidades encontradas sólo se presentan en escalas muy pequeñas, 
fluctuaciones que contrastan con la densidad de fondo.  
Como consecuencia se han enunciado dos postulados que permiten  una mayor 
simplicidad en el estudio del cosmos, el principio cosmológico y el postulado 
de Weyl. 


\

Principo cosmol\'ogico: \emph{El universo es isotr\'opico y homog\'eneo 
en grandes escalas.}

\

No sobra hacer un poco de claridad sobre los términos usados, homogéneo se 
refiere a que independientemente de donde ubiquemos el sistema de referencia se 
observará la misma estructura o propiedades del Universo. Por su parte, la
isotropía establece que independiente de la dirección en que se realice una 
observación se deben nuevamente observar las mismas propiedades del Universo. 
En otras palabras, se tiene simetría rotacional y traslacional para el sistema
de referencia escogido. 

En la actualidad dichas características son observables en escalas de mega parsecs
pero ya que el Universo se encuentra en expansión, dicha escala claramente 
va a depender de la época cósmica en partícular. 


Además existe otra premisa importante a tener en cuenta en un módelo 
cosmológico, está es la expansión del Universo, la cuál deja fuertes 
consecuencias en la predicción de la evolución de éste, 
que dependerán del contenido de masa y energía total en el Universo. 
Anteriormente se tenía una idea arraigada, que el universo era estático, muestra de 
ello es el módelo cosmológico propuesto por Einstein donde se incluía 
una constante tal que se satisfacíera dicha condición.
Pero fue gracias a observaciones de galaxias cercanas realizadas 
por Edwin Hubble, que se concluyó que las galaxias en su mayoría 
tienen un corrimiento al rojo, en otras palabras, se están alejando 
de nosotros. 
\

Una métricaque satisface las condiciones de isotropía y homogeneidad 

\begin{equation}
ds^2= c^2dt^2-a(t)^2\left[\frac{d^2r}{1-Kr^2} +r^2(d^2\theta
 + \sin^2\theta d^2\phi )\right]
\label{metrica}
\end{equation} 	

es la métrica Robertson Walker. El término $a(t)$ 
es el factor de escala, describe como la distancia relativa entre 
dos observadores fundamentales cambia con el tiempo y $K$ es la 
constante de curvatura en el tiempo actual, define la geometría
del Universo. 
 
%**********************************************************************************************************************
\begin{figure}[htbp]
       \centering
               \includegraphics[width=0.4\textwidth]{Images/factordeescala.pdf}
       \caption{ \small Factor de Escala en función del tiempo. La expansión del Universo para 
       diferentes contribuciones a la densidad, se obtiene un Universo cerrado para $\Omega_m = \Omega_o>1$,
       los parámetros WMAP7 muestran que el Universo sufre una expansión acelerada. 
       }
       \label{factor}
 \end{figure}
%**********************************************************************************************************************


A grandes escalas la interacción fundamental de mayor importancia es 
la gravitacional, por lo que la teoría general de la relatividad 
es una herramienta esencial en el estudio del cosmos. 
Anteriormente se consideraba válida la teoría newtoniana de la gravitación pero 
difiere considerablemente al compararla con la relatividad, 
puesto que el tiempo y el espacio dejan de ser entes absolutos 
además de ser afectados por 
el contenido de energ\'ia y materia presente en el universo. 

En el marco de la teoría newtoniana, la ecuaci\'on de Poisson ofrece una 
relaci\'on entre la segunda derivada del campo y la densidad de materia 
que es la fuente de dicho campo

\[ \nabla^2\Phi=4\pi G\rho\]

a ésta se reduce la ecuación de campo bajo las condiciones de bajas velocidades 
y campo gravitacional débil ($\Phi/c^2<< 1$). 
Pero la ecuación de campo no solo incluye la de Poisson sino además todo lo 
relacionado con la dinámica newtoniana. La ecuación de campo Hilbert-Einstein es 
entonces

\eq{
R_{\mu\nu}-\f{1}{2}g_{\mu\nu}R-g_{\mu\nu}\Lambda = \f{8\pi G}{c^4}T_{\mu\nu}
}

ésta es una ecuaci\'on tensorial de 6 componentes independientes. 
El primer t\'ermino de la izquierda corresponde al tensor de Ricci 
o a segundas derivadas del tensor m\'etrico $g_{\mu\nu}$. 
En el segundo t\'ermino se encuentra el escalar de curvatura 
que define la geometr\'ia. 
En el tercer t\'ermino $\Lambda$ es la constante cosmol\'ogica, 
es asociada a la la densidad con la que contribuye el vacío a 
la densidad total y sería responsable por la expansión acelerada del 
Universo. 
Al lado derecho de la ecuación se encuentra el tensor momentun-energ\'ia 
en la que se incluyen todas las contribuciones de energía y momentum 
como su nombre lo indica.

Es decir que al lado izquierdo se encuentran los t\'erminos que dan 
cuenta por la geometr\'ia del universo mientras que en la derecha 
los relacionados con la distribuci\'on de materia y e\-ner\-gía. 
Por consiguiente se podría afirmar que la geometr\'ia es determinada por 
el contenido de materia-energ\'ia del universo, aunque estrictamente 
hablando el tensor de energ\'ia momentum tambi\'en depende del tensor m\'etrico. 


\

Existen diversas soluciones a la ecuación de Einstein 
pero no muchas en forma analítica, por ejemplo Schwarzschild encontró 
la m\'etrica de un astro est\'atico y con simetría esférica. 
Otra soluci\'on es la m\'etrica de Kerr que corresponde a un astro en 
rotaci\'on con un campo estacionario. 
Claramente la m\'etrica de Robertson-Walker también satisface dichas ecuaciones.


A partir de las ecuación de campo de Einstein y la métrica Robertson-Walker 
es posible proponer módelos cosmológicos que den cuenta por la dinámica observada 
en el Universo. 

Una forma muy usada de las ecuaciones de Friedmann se obtiene de

\begin{equation}
\frac{H^2(z)}{H_o^{2}}=\Omega_{m,o}\left(1+z\right)^3+
\Omega_{r,o}\left(1+z\right)^4+ \Omega_{\Lambda,o} + ( 1-\Omega_o)
\left(1+z\right)
\label{fried2}
\end{equation}

donde $\Omega_o=\Omega_{m,o} +\Omega_{r,o}+\Omega_{\Lambda,o}$ ,
se ha introducido la relación entre el redshift y el factor de escala 
$1+z=1/a$. Se observan las diferentes contribuciones de la densidad al 
parámetro de Hubble, esto es, la densidad de materia, radiación y vacío. 
Cada una contribuye dependiendo de la expansión del Universo, aunque 
la energía del vacío no varía con el redshift.

\

Inicialmente el Universo estaba dominado por radiación, 
durante esta época la radiación estaba acoplada con la materia, 
es decir, la longitud de onda de De Broglie de los electrones es 
comparable a la longitud de onda de la radiación. Por lo anterior 
el camino libre medio de los fotones era despreciable, ocasionando
que el universo fuera opaco. 
Durante éste acople, la temperatura de la radiación es igual a 
la de la materia y corresponde a la de un cuerpo negro. 

Tal como se ve en la grafica \ref{densidad}, a partir de $z=3230$ la materia
se vuelve la mayor contribuci\'on a la densidad del universo y
cuando $z=1100$ la temperatura ha descendido lo suficiente para que 
la tasa de recombinación sea mayor que la de ionización. 
En este caso recombinación se refiere a la formación de átomos neutros, 
lo que permitió el desacople materia radiación.
Pero la última dispersión de la radiación debido a la 
materia aún puede observarse, corresponde a la radiación 
cósmica de fondo, la cual como consecuencia de la expansión del universo 
ha ido enfriandose hasta alcanzar $T$ = $2.7K$. 
  
%**********************************************************************************************************************
\begin{figure}[htbp]
       \centering
               \includegraphics[width=0.4\textwidth]{Images/density.pdf}
       \caption{ \small Dependencia en el redshift para $\Omega_\Lambda$, $\Omega_m$ y $\Omega_r$.
       A partir de $z_m$ la densidad de materia domina hasta $z_\Lambda$ 
       donde el término de radiación empieza a dominar. El desacople radiación materia 
       se da para $z_{rec}$. }
       \label{densidad}
 \end{figure}
%**********************************************************************************************************************

En la actualidad el Universo esta dominado por la densidad de vacío, aunque ésta 
es constante puesto que no depende del factor de escala $\rho_{\Lambda}=-c^4\Lambda/8\pi G$,
en contraposición con la materia que depende como $a^{-3}$ y la radiación como $a^{-4}$ 
haciendo que dichas contribuciones disminuyan con el tiempo. 
La constante cosmológica que se observa en esta definición puede ser asociada a una fuerza 
repulsiva que se opone a la gravedad lo que podría dar cuenta de la expansi\'on 
acelerada del universo. 

	
%&&&&&&&&&&&&&&&&&&&&&&&&&&&&&&&&&&&&&&&&&&&&&&&&&&&&&&&&&&&&&&&&&&&&&&&&&&&&&&&&&&&&&&&&&&&&&&&&&&&&&&&&&&&&&&&&&&&&&&
\subsection*{Evolución de las Perturbaciones de Densidad en Régimen Newtoniano}
%&&&&&&&&&&&&&&&&&&&&&&&&&&&&&&&&&&&&&&&&&&&&&&&&&&&&&&&&&&&&&&&&&&&&&&&&&&&&&&&&&&&&&&&&&&&&&&&&&&&&&&&&&&&&&&&&&&&&&&

Tal como se mencionó previamente no podemos detectar radiación proveniente 
de una época previa a la de reionización, a causa de la dispersión Compton
que mantenía acopladas la radiación y materia.
Aunque si se puede observar la distribución altamente homogénea de la materia
a este redshift en la radiación cósmica de fondo (Figura \ref{CMB}\footnote{
Imagen WMAP obtenida en \url{http://lambda.gsfc.nasa.gov/product/map/current/m_images.cfm}}).  
Dicha radiación cae en el microondas y da cuenta de variaciones 
en la densidad de fondo, estás últimas resultan ser las causantes de 
la estructura del Universo observada a más pequeñas escalas.
Las variaciones en la densidad son fluctuaciones, $\delta$, que fueron 
incrementando en el transcurso del tiempo, al menos las de nuestro
interes. 
Fue hasta que $\delta\sim 1$  que las fluctuaciones fueron lo 
suficientemente grandes para ser consideradas objetos indi\-vi\-dua\-les, es decir,
su movimiento no solo se debía al flujo de Hubble. 
Lo anterior permite dar una cota superior en redshift a la formación de galaxias, 
que se encuentra alrededor de $z\sim 100$, cuando la densidad promedio de estás 
comparada con las de fondo es aproximadamente $1\times 10^6$. 

Las fluctuaciones iniciales pueden ser tratadas en un regimen líneal mientras 
los contrastes en densidad sean $\delta\ll 1$, por lo que para $z<100$ es una 
suposición razonable. 

%######################################################################################################################
\subsubsection*{Descripción Newtoniana}
%######################################################################################################################

Las fluctuaciones de densidad iniciales tienen una longitud característica mucho menor 
que la distancia de Hubble, la última se define como $d_H \approx ct$. En otras palabras,
el tamaño de las fluctuaciones es muy pequeño comparada con la escala en la que la 
curvatura del Universo es signficativa, permitiendo que la aproximación newtoniana sea válida. 


Como se busca estudiar las fluctuaciones de densidad es útil expresar la 
densidad como $\rho = \overline{\rho}+\delta\rho$, donde $\overline{\rho}$
es la densidad de fondo. Adicionalmente es necesario aclarar que la velocidad 
con la que las partículas se desplazan corresponde a dos contribuciones diferentes, 
la primera es causada por la expansión del Universo y la 
otra es la velocidad propia de la partícula. Partiendo de esto, se
podría pensar en cambiar el sistema de referencia
de las ecuaciones \ref{euler} tal que se satisfaga una descripción lagrangiana, 
esto es, moverse con la expansión del Universo. 
Veamos esto en más detalle, la velocidad en la descripción euleriana está dada 
por $\textbf{u}= a\dot{\textbf{x}}+
\textbf{x}\dot{a} = \textbf{v}+\textbf{x}\dot{a}$, donde $\textbf{v}$ es la velocidad 
peculiar de la partícula y $\textbf{x}\dot{a}$ es la velocidad de expansión del Universo.

Se puede obtener una ecuación de onda para las fluctuaciones de densidad 

\begin{equation}
\f{\partial^2\delta}{\partial t^2}+2\f{\dot{a}}{a}\pder{\delta}{t} =
4\pi G \overline{\rho}\delta + \f{C_s^2}{a^2}\nabla^2\delta+\f{2}{3}\f{\overline{T}}{a^2}\nabla^2s
\label{onda}
\end{equation}

donde $\overline{T}$ es la temperatura del fondo, $C_s$ es la velocidad del sonido. 
Al lado derecho se encuentran las fuentes de las 
fluctuaciones de densidad, como el campo gravitacional, la curvatura dada en términos de la segunda 
derivada de las perturbaciones y cambios en la entropía del sistema. En el lado izquierdo 
se encuentra el parámetro de Hubble que responde por la disipación de la fluctuación debido 
a la expansión del Universo. 

Se propone una solución a la ecuación de perturbaciones en términos de la serie de Fourier 

\begin{eqnarray}
\delta(x,t) &=& \sum_k \delta_k(t)e^{ik\cdot x} \nonumber\\
s(x,t) &=& \sum_k s(t)e^{ik\cdot x} \nonumber
\end{eqnarray}

otro aspecto a tener presente es la independencia de las funciones $e^{ik\cdot x}$ lo que permite expresar 
la ecuación \ref{onda} como

\begin{equation}
\f{d^2\delta_k(t)}{dt^2}+2\f{\dot{a}}{a}\der{\delta_k(t)}{t} = 
\left[4\pi G\overline{\rho}-\f{C_s^2 k^2}{a^2}\right]\delta_k(t)
-\f{2}{3}\f{\overline{T}}{a^2}k^2s_k(t)
\label{ec.modos}
\end{equation}

la solución de está ecuación nos da los coeficientes de expansión de la serie de 
Fourier, obteniendo así el comportamiento de las fluctuaciones de densidad durante
el tiempo en que el régimen newtoniano permanece válido. 


%######################################################################################################################
\subsubsection*{Espectro de Potencias}
%######################################################################################################################

Para realizar un estudio del campo de densidad es necesario realizar un 
tratamiento estadístico que nos permita conocer las propiedades de éste. 
En esta dirección, se puede asumir que las fluctuaciones $\delta$ siguen una
distribución normal centrada en $\langle \delta \rangle = 0$, lo cual es soportado
por escenarios inflacionarios que predicen la formación de fluctuaciones como un
campo gaussiano. Entonces, la probabilidad de que 
se tenga un campo de densidad, o dicho en otra forma, la probabilidad de que se tenga 
una distribución específica de fluctuaciones de densidad en el espacio de fourier está
dada por 

\begin{equation}
\mathcal{P}(\delta_{\mbox{\boldmath$\kappa$}})r_{\mbox{\boldmath$\kappa$}}dr_{\mbox{\boldmath$\kappa$}}d\phi_{\mbox{\boldmath$\kappa$}}
=
\exp\left[-\f{r_{\mbox{\boldmath$\kappa$}}^2}{2V_u^{-1}P(\kappa)}\right]	\f{r_{\mbox{\boldmath$\kappa$}}}{V_u^{-1}}
\f{dr_{\mbox{\boldmath$\kappa$}}}{P(\kappa)}\f{d\phi_{\mbox{\boldmath$\kappa$}}}{2\pi}
\label{probabilidad}
\end{equation}

donde los términos dependientes de $r_{\mbox{\boldmath$\kappa$}}$ corresponden a la amplitud de las perturbaciones 
y los de $\phi_{\mbox{\boldmath$\kappa$}}$ a la fase, la última varía aleatoriamente entre $[0,2\pi)$. 
Esta función de densidad de probabilidad conjunta es de utilidad porqué permite independencia en los términos 
$\delta_{\mbox{\boldmath$\kappa$}}$, es decir que es el producto de cada modo 

\[
\mathcal{P}_{\mbox{\boldmath$\kappa$}}(\delta_{\mbox{\boldmath$\kappa$}1},...,\delta_{\mbox{\boldmath$\kappa$}N})=
\prod_{\mbox{\boldmath$\kappa$}}\mathcal{P}_{\mbox{\boldmath$\kappa$}}(\delta_{\mbox{\boldmath$\kappa$}})
\]

lo que no sucede al aplicar la transformada inversa de fourier ya que la función de probabilidad no es separable en el espacio de
coordenadas. El término $P(\kappa)$ en \ref{probabilidad} es el espectro de potencias, éste es definido en el 
espacio de fourier y está relacionada con la función de correlación en el espacio real

\[
P(k) = 4\pi\int_0^\infty \mathcal{E} (r)sinc(kr)r^2dr = V_u  \langle|\delta_{\mbox{\boldmath$\kappa$}}|^2\rangle
\]

$\mathcal{E} (r)$ proporciona la correlación entre dos puntos en el espacio. Adicionalmente se ha encontrado
que el espectro de potencias esperado por la teoría inflacionaría es $P(\kappa)= k^n$, si $n$ toma
el valor de 1 recibe el nombre de espectro Harrison-Zeldovich, es el que mejor 
resultados proporciona. La isotropía del Universo es tomada en cuenta
para el espectro de potencias puesto que se promedia sobre todas las posibles orientaciones del vector 
${\mbox{\boldmath$\kappa$}}$, adicionalmente debe normalizarse con la cantidad $\sigma_8$ que da cuenta
por la amplitud de las fluctuaciones a $8$Mpc/h. 
Cuando el campo se asume como gaussiano se puede mostrar 
que el espectro de potencias contiene toda la información del campo, por lo que
para calcularlo se encuentra el espectro de potencias,
obteniendo la propabilidad de \ref{probabilidad}. Al último se le aplica la transformada inversa
obteniendo así la distribución del campo en coordenadas espaciales. Dicho campo de densidad inicial es usado
para simulaciones cosmológicas o también puede ser tomado de las observaciones realizadas
de la radiación cósmica de fondo. 

%==============================================================================
\section{Objectives}
%==============================================================================

\subsection*{General Objective}


To study the properties of the baryon acoustic oscillations (BAOs), amplitude and width, using as
tracers the distribution of the halos in the sloan digital sky survey and their dependence with the tracer
halo population.

\subsection*{Specific Objectives}

\begin{itemize} 

\item[-] Study the physics of BAOs and the bases necessary to understand
the expected behaviour of BAOs. 

\item[-] Use observations and baryonic  and  dark matter numerical simulations to construct the power 
spectrum and this way obtain the BAOs.  

\item[-] Analyse biases and possible corrections to the power spectrum that can affect the baryonic
acoustic oscillations obtained.

\item[-] Determine in which way the structure scale is related to the amplitude and width of the baryon
acoustic oscillations. 

\item[-] Find if there is a change in the BAOs position with the structure scale and quantify it. 

\item[-] Find a possible correlation between the BAOs properties and structure formation. 

\item[-] Write a master thesis.

\end{itemize}


%==============================================================================
\section{Methodology}
%==============================================================================

The work consists in analysing the behaviour of BAOs in different scenarios, then it is fundamental
to find a way to obtain the BAOs either, from simulations or observations. For this reason, 
it becomes necessary to develop a numerical tool to find such BAOs in any case. But in order to 
build such a tool it is obviously needed a physical model that can account for BAOs. 
Using the numerical tool there are going to be done several realizations, this is done seeking to avoid errors
associated to cosmic variance and shot noise. Finally after this is done, there is going to be an 
analysis of the BAOs in simulation(theoretical model) against the BAOs recovered from observations.

\

A more detailed procedure is shown below.

\begin{itemize}

\item[-] \textbf{ Bibliographical review }
	
In this first stage it will look for books, reviews and articles related with the physics of 
BAOs, the different ways to find BAOs through the field density or correlation function, the 
power spectrum, power spectrum calculation and corrections associated. As well as a search 
through data base, specially to look for observational data  from the SDSS, DR-7 and DR-12 
that are going to be used.

\item[-] \textbf{ Construction of models } 

Using the information collected in the bibliographical review
it will be chosen a model to construct the density field of a baryonic  and  dark matter simulation,
the power spectrum and the corrections to take into account. Equally, for the observational
data collected is going to be used an specific model to construct the dark matter halos
from which  BAOs are going to be found.

\item[-] \textbf{ Model implementation } 

In this section it will be coded the models proposed, i.e. ,
a model to obtain BAOs from N body cosmological simulations and BAOs from certain observations.  
The code will allow to make several numerical realizations of different baryonic  and  dark matter
simulations and compare them with the BAOs obtained from observations. 

\item[-] \textbf{ Analysis and conclusions } 

It is necessary to analyse the results obtained in the 
previous sesion. One of the most important 
issue is to characterize the BAOs through this results and show if there are changes in its properties  
with the cosmological scale, specifically if there is any possible change in the position or 
width of the BAOs obtained in the power spectrum.  

\item[-] \textbf{ Thesis and paper } 

The results and conclusions will be written in  a master thesis and 
an article. 

\end{itemize}

%==============================================================================
\section{Expected Results}
%==============================================================================

The expected results when finishing the master studies are 

\begin{itemize}

\item[-] A code that finds the BAOs of a baryonic  and  dark matter cosmological simulation, that also includes 
corrections to the power spectrum that possibly affect the BAOs results. 

\item[-] Obtain BAOs from several observations, SDSS, DR-7 and DR-12, and conclude what changes appear 
in the width and position of the BAOs with the structure scale. 

\item[-] Comparison between the results obtained with baryonic  and  dark matter cosmological simulations
and the BAOs obtained from observations to determine in an optimal way the properties of the BAOs.  

\item[-] Study the dependence of tracer bias on the BAOs properties measured in the observations. 

\item[-] A master thesis with the results obtained. 

\item[-] A submitted article with the results obtained.  

\end{itemize}

%==============================================================================
\section{Scientific Impact}
%==============================================================================

As it was pointed before, there was a time where matter and radiation was coupled, this caused acoustic
oscillations due to the competing forces, radiation pressure and the gravity. Then, when decoupling occurred 
a characteristic scale imprint appeared to various scales. Only the scales of around $150$Mpc survived 
but these ones can be used as standard rulers  \cite{Eisenstein98}. Specifically these standard rulers 
are more reliable  at high redshifts\cite{Wagner08} than other ones. 

\

A very active investigation field is dark energy (DE),  not only is about $70\%$ of the content of energy and
matter of the universe but it has not been clearly understood nowadays. A equation of state (EOS) of the dark
energy is then a relevant issue to be studied and in general it depends on the redshift, i.e., $w(z)$. Then
it becomes necessary to have standard rulers for different stages, including high redshifts. Therefore, the 
BAOs allows to get tight constraints on the parameters of the DE EOS. 

\

Hence, the relevance of this work is study the properties of the BAOs and this way to provide a better understanding
of their behaviour, specifically to observe what are the changes in their position and width with scale structure
getting a profound insight in the physics of BAOs. 


%==============================================================================
\section{Schedule}
%==============================================================================
Next it is shown a table with the proposed activities scheduled for each term.

\begin{table}[h]
\begin{flushleft}
\begin{center}
  \begin{tabular}{l  c c c c } \hline\hline
	\centering\textbf{Goals} & \textbf{Term I} & \textbf{Term II} & 
	\textbf{Term III} & \textbf{Term IV} \\ \hline\hline
	
	 Bibliographical review & X & & & \\
	 Construction of models & X & X & & \\
	 Model implementation &  & X & X & \\
	 Analysis and conclusions &  &  & X & \\
	 Thesis and paper &  &  &  & X \\
	\hline\hline
  \end{tabular}  
  \caption{ Terms range from 2015-02 for term I, up to 2017-01 for term IV.}
\end{center}
\end{flushleft}
\end{table}

%==============================================================================
\bibliographystyle{latex/mn2e}
\renewcommand{\bibname}{11\ \ \ \ Bibliography}
\small
\bibliography{references.bib}
%==============================================================================


\end{document}